\section{INTRODUCCI\'ON}
\label{sec:intro}
Muchos de los problemas de ingenier\'ia puede ser relacionados o modelados con una b\'usqueda a trav\'es de grafos, la b\'usqueda es uno de estos y que ha sido ampliamente estudiado generando dos grupos de acuerdo a la forma en que hallan la soluci\'on.

Estos grupos son denominados b\'squeda informada y no-informada, en los que se encuentran diferentes algoritmos. En la b\'usqueda no-informada Amplitud, Profundidad y Costo Uniforme y en la b\'usqueda informada Avara o heuristica y A* (A estrella), que son discutidos y analizados en este informe.

El problema presentado en este informe es la p\'erdida de tres de los personajes de la pel\'icula Buscando a Nemo \cite{wiki:nemo}, ``Nemo, Marlin y Dori'' en un arrecife y para ayudar a resolver este problema se han implementado diferentes estrategias de b\'usqueda aplicando IA, a trav\'es de un nuevo personaje Robot, este es guiado por cada uno de los algoritmos y debe hallar los tres personajes en un orden especifico. El arrecife se carga desde un archivo de texto plano y a partir de este el robot debe ``construir'' la ruta pasando por una serie de obst\'aculos y ayudas que este posee.

Por esta razon las pruebas consisten en un conjunto de archivos que moldean arrecifes de diferentes tama\~nos y en los que se ejecuta cada uno de las implementaciones de algoritmos y de acuerdo a las salidas se obtiene las rutas, el costo de la ruta y los tiempo de ejecuci\'on, que son comparados y finalmente permiten concluir que una ``buena'' heurística, en concordancia a lo visto en clase, puede conducir a un mejor resultado en cuanto a tiempo de ejecuci\'on, espacio en memoria y costo de la b\'usqueda.\\

Este informe primero describe cada uno de los algoritmos, la forma de implementaci\'on, en la segunda secci\'on indica el tama\~no algunas de las entradas de prueba, en la tercera secci\'on el uso de la aplicaci\'on y la codificaci\'on, en la cuarta secci\'on algunos de los resultados obtenidos y finalmente la discusi\'on de los resultados y las conclusiones.
 